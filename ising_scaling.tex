\documentclass[
  aps,
  prb,
  reprint,
  longbibliography,
  floatfix
]{revtex4-2}

\usepackage[utf8]{inputenc}
\usepackage[T1]{fontenc}
\usepackage{newtxtext, newtxmath}
\usepackage[
  colorlinks=true,
  urlcolor=purple,
  citecolor=purple,
  filecolor=purple,
  linkcolor=purple
]{hyperref}
\usepackage{amsmath}
\usepackage{graphicx}
\usepackage{xcolor}

\begin{document}

\title{Smooth Ising universal scaling functions}

\author{Jaron Kent-Dobias}
\affiliation{Laboratoire de Physique de l'Ecole Normale Supérieure, Paris, France}

\author{James P.~Sethna}
\affiliation{Laboratory of Atomic and Solid State Physics, Cornell University, Ithaca, NY, USA}

\date\today

\begin{abstract}
\end{abstract}

\maketitle

At continuous phase transitions the thermodynamic properties of physical
systems have singularities. Celebrated renormalization group analyses imply
that not only the principal divergence but also entire additive functions are
\emph{universal}, meaning that they will appear at any critical points that
connect phases of the same symmetries in the same spatial dimension. The study
of these universal functions is therefore doubly fruitful: it provides both a
description of the physical or model system at hand, and \emph{every other
system} whose symmetries, interaction range, and dimension puts it in the same
universality class.

The continuous phase transition in the two-dimensional Ising model is perhaps
the most well studied, and its universal thermodynamic functions have likewise
received the most attention. Precision numeric work both on the lattice
critical theory and on the ``Ising'' critical field theory (related by
universality) have yielded high-order polynomial expansions of those functions
in various limits, along with a comprehensive understanding of their analytic
properties and even their full form \cite{Fonseca_2003_Ising, Mangazeev_2008_Variational, Mangazeev_2010_Scaling}. In parallel, smooth approximations of the
Ising ``equation of state'' have produced convenient, evaluable, differentiable
empirical functions \cite{Guida_1997_3D, Campostrini_2000_Critical, Caselle_2001_The}. Despite being differentiable, these approximations become
increasingly poor when derivatives are taken due to the presence of a subtle
essential singularity [refs] that is previously unaccounted for.

This paper attempts to find the best of both worlds: a smooth approximate
universal thermodynamic function that respects the global analyticity of the
Ising free energy, for both the two-dimensional Ising model (where much is
known) and the three-dimensional Ising model (where comparatively less is
known). First, parametric coordinates are introduced that remove unnecessary
nonanalyticities from the scaling function. {\bf [The universal scaling function has the nonanalyticities. You are writing it as a function with the right singularity, modulated somehow with an analytic function.]} Then the arbitrary analytic
functions that compose those coordinates are approximated by truncated
polynomials whose coefficients are fixed by matching the series expansions of
the universal function in three critical regimes: at no field and low
temperature, no field and high temperature, and along the critical isotherm.

This paper is divided into four parts. First, general aspects of the problem
will be reviewed that are relevant in all dimensions. Then, the process
described above will be applied to the two- and three-dimensional Ising models.

\section{General aspects}

\subsection{Universal scaling functions}

Renormalization group analysis of the Ising critical point indicates that the free energy per site $f$ may be written, as a function of the reduced temperature $t=(T-T_c)/T_c$ and external field $h=H/T$,
\begin{equation}
\label{eq:AnalyticSingular}
  f(t,h)=g(t,h)+f_s(t,h)
\end{equation}
with $g$ a nonuniversal analytic function that depends entirely on the system
in question and $f_s$ a singular function. The singular part $f_s$ can be said
to be universal in the following sense: for any system that shares the
universality with the Ising model, if the near-identity smooth change of coordinates
$u_t(t, h)$ and $u_h(t,h)$ is made such that the flow equations for the new
coordinates are exactly linearized, e.g.,
\begin{align} \label{eq:flow}
  \frac{du_t}{d\ell}=\frac1\nu u_t
  &&
  \frac{du_h}{d\ell}=\frac{\beta\delta}\nu u_h,
\end{align}
{\bf [I've been wondering for some time about eqn (1) and the flow equation for $df/d\ell$. If $df/d\ell = D f +$ [arbitrary stuff involving f, t, and s], what arbitrary stuff is allowed in order for eqn~\ref{eq:AnalyticSingular} to hold?] }
then $f_s(u_t, u_h)$ will be the same function, up to constant rescalings of
the free energy and the nonlinear scaling fields $u_t$ and $u_h$. In order to
fix this last degree of freedom {\bf [the two rescalings?]}, we adopt the same convention as used by
\cite{Fonseca_2003_Ising}. The dependence of the nonlinear scaling variables on
the parameters $t$ and $h$ is also system-dependent, and their form can be
found for common model systems (the square- and triangular-lattice Ising
models) in the literature \cite{Clement_2019_Respect}.

With the flow equations \eqref{eq:flow} along with that for the free energy,
the form of $f_s$ is highly constrained, further reduced to a universal
\emph{scaling function} of a single variable $u_h|u_t|^{-\beta\delta}$ (or equivalently
$u_tu_h^{-1/\beta\delta}$) with multiplicative power laws in $u_t$ or $u_h$ and
(sometimes) simple additive singular functions of $u_t$ and $u_h$. The special
variables are known as scaling invariants, as they are invariant under the flow
\eqref{eq:flow}. Reasonable assumptions about the analyticity of the scaling
function of a single variable then fixes the principal singularity at the
critical point.

\subsection{Essential singularities and droplets}

Another, more subtle, singularity exists which cannot be captured by the
multiplicative factors or additive terms, residing instead inside the scaling
function itself. The origin can be schematically understood to arise from a
singularity that exists in the complex free energy of the metastable phase of
the model, suitably continued into the equilibrium phase. When the equilibrium
Ising model with positive magnetization is subjected to a small negative
magnetic field, its equilibrium state instantly becomes one with a negative
magnetization. However, under physical dynamics it takes time to arrive at this
state, which happens after a fluctuation containing a sufficiently large
equilibrium `bubble' occurs.

The bulk of such a bubble of radius $R$ lowers the free energy by
$2M|H|V_dR^d$, where $d$ is the dimension of space, $M$ is the magnetization,
$H$ is the external field, and $V_d$ is the volume of a $d$-ball, but its
surface raises the free energy by $\sigma S_dR^{d-1}$, where $\sigma$ is the
surface tension between the stable--metastable interface and $S_d$ is the
volume of a $(d-1)$-sphere. The bubble is sufficiently large to decay
metastable state when the differential bulk savings outweigh the surface costs.

This critical bubble occurs with free energy cost
\begin{equation}
  \begin{aligned}
    \Delta F_c
      &\simeq\left(\frac{S_d\sigma}d\right)^d\left(\frac{d-1}{2V_dM|H|}\right)^{d-1} \\
      &\simeq T\left(\frac{S_d\mathcal S(0)}d\right)^d\left[\frac{2V_d\mathcal M(0)}{d-1}ht^{-\beta\delta}\right]^{-(d-1)}
  \end{aligned}
\end{equation}
where $\mathcal S(0)$ and $\mathcal M(0)$ are the critical amplitudes for the
surface tension and magnetization, respectively \textbf{[find more standard
notation]} \cite{Kent-Dobias_2020_Novel}.
In the context of statistical mechanics, Langer demonstrated that the decay rate is asymptotically proportional to the imaginary part of the free energy in the metastable phase, with (assuming Arrhenius behavior)
\begin{equation}
  \operatorname{Im}f\propto\Gamma\sim e^{-\beta\Delta F_c}=e^{-1/(B|h||t|^{-\beta\delta})^{d-1}}
\end{equation}
which can be more rigorously related in the context of quantum field theory [ref?].
  
This is a singular contribution that depends principally on the scaling
invariant $ht^{-\beta\delta}\simeq u_h|u_t|^{-\beta\delta}$. It is therefore
suggestive that this should be considered a part of the singular free energy
$f_s$, and moreover part of the scaling function that composes it. We will therefore make the ansatz that
\begin{equation}
  \operatorname{Im}\mathcal F_-(\xi)=A\Theta(-\xi)|\xi|^{-b}e^{-1/(B|\xi|)^{d-1}}\left(1+O(\xi)\right)
\end{equation}
\cite{Houghton_1980_The}
The exponent $b$ depends on dimension and can be found through a more careful
accounting of the entropy of long-wavelength fluctuations in the droplet
surface \cite{Gunther_1980_Goldstone}.
Kramers--Kronig type dispersion relations can then be used to recover the
singular part of the real scaling function from this asymptotic form.

\subsection{Schofield coordinates}

The invariant combinations $u_h|u_t|^{-\beta\delta}$ or
$u_t|u_h|^{-1/\beta\delta}$ are natural variables to describe the scaling
functions, but prove unwieldy when attempting to make smooth approximations.
This is because, when defined in terms of these variables, scaling functions
that have polynomial expansions at small argument have nonpolynomial expansions
at large argument. Rather than deal with the creative challenge of dreaming up
functions with different asymptotic expansions in different limits, we adopt
different coordinates, in terms of which a scaling function can be defined that
has polynomial expansions in \emph{all} limits.

In all dimensions, the Schofield coordinates $R$ and $\theta$ will be implicitly defined by
\begin{align} \label{eq:schofield}
  u_t(R, \theta) = Rt(\theta)
  &&
  u_h(R, \theta) = R^{\beta\delta}h(\theta)
\end{align}
where $t$ and $h$ are polynomial functions selected so as to associate different scaling limits with different values of $\theta$. We will adopt standard forms for these functions, given by
\begin{align} \label{eq:schofield.funcs}
  t(\theta)=1-\theta^2
  &&
  h(\theta)=\left(1-\frac{\theta^2}{\theta_c^2}\right)\sum_{i=0}^\infty h_i\theta^{2i+1}
\end{align}
This means that $\theta=0$ corresponds to the high-temperature zero-field line,
$\theta=1$ to the critical isotherm at nonzero field, and $\theta=\theta_c$ to
the low-temperature zero-field (phase coexistence) line.

In practice the infinite series in \eqref{eq:schofield.funcs} cannot be
entirely fixed, and it will be truncated at finite order. We will notate the
truncation an upper bound of $n$ by $h^{(n)}$. The convergence of the
coefficients as $n$ is increased will be part of our assessment of the success
of the convergence of the scaling form.

\section{The 2D Ising model}

\subsection{Definition of functions}

The scaling function for the two-dimensional Ising model is the most
exhaustively studied universal forms in statistical physics and quantum field
theory.
\begin{equation} \label{eq:free.energy.2d.low}
  f_s(u_t, u_h)
  = |u_t|^2\mathcal F_{\pm}(u_h|u_t|^{-\beta\delta})
    +\frac{u_t^2}{8\pi}\log u_t^2
\end{equation}
where the functions $\mathcal F_\pm$ have expansions in nonnegative integer powers of their arguments.
\begin{equation} \label{eq:free.energy.2d.mid}
  f_s(u_t, u_h)
  = |u_h|^{2/\beta\delta}\mathcal F_0(u_t|u_h|^{-1/\beta\delta})
    +\frac{u_t^2}{8\pi}\log u_h^{2/\beta\delta}
\end{equation}
where the function $\mathcal F_0$ has a convergent expansion in nonnegative integer powers of its argument.
To connect with Mangazeev and Fonseca, $\mathcal F_0(x)=\tilde\Phi(-x)=\Phi(-x)+(x^2/8\pi) \log x^2$ and $\mathcal F_\pm(x)=G_{\mathrm{high}/\mathrm{low}}(x)$.

Schofield coordinates allow us to define a global scaling function $\mathcal F$ by
\begin{equation} \label{eq:schofield.2d.free.energy}
  f_s(R, \theta) = R^2\mathcal F(\theta) + t(\theta)^2\frac{R^2}{8\pi}\log R^2
\end{equation}
The scaling function $\mathcal F$ can be defined in terms of the more
conventional ones above by substituting \eqref{eq:schofield} into \eqref{eq:free.energy.2d.low} and
\eqref{eq:free.energy.2d.mid}, yielding
\begin{equation} \label{eq:scaling.function.equivalences.2d}
  \begin{aligned}
    &\mathcal F(\theta)
    =t(\theta)^2\mathcal F_\pm\left[h(\theta)|t(\theta)|^{-\beta\delta}\right]
      +\frac{t(\theta)^2}{8\pi}\log t(\theta)^2 \\
    &=|h(\theta)|^{2/\beta\delta}\mathcal F_0\left[t(\theta)|h(\theta)|^{-1/\beta\delta}\right]
      +\frac{t(\theta)^2}{8\pi}\log h(\theta)^{2/\beta\delta}
  \end{aligned}
\end{equation}
Examination of \eqref{eq:scaling.function.equivalences.2d} finds that $\mathcal F$ has expansions in integer powers in the entire domain $-\theta_c\leq0\leq\theta_c$.


\begin{equation} \label{eq:im.f.func.2d}
  f(x)=\Theta(-x) |x| e^{-1/|x|}
\end{equation}
\begin{equation}
  \operatorname{Im}\mathcal F(\theta)=A\left\{f\left[\tilde B(\theta_c-\theta)\right]+f\left[b(\theta_c+\theta)\right]\right\}
\end{equation}

\begin{equation}
  \begin{aligned}
    \operatorname{Re}\mathcal F(\theta)
      &=G(\theta^2)-\frac{\theta^2}\pi\int d\vartheta\, \frac{\operatorname{Im}\mathcal F(\vartheta)}{\vartheta^2(\vartheta-\theta)} \\
      &=G(\theta^2)+\frac A\pi\left\{f[\tilde B(\theta_c-\theta)]+f[\tilde B(\theta_c+\theta)]\right\}
  \end{aligned}
\end{equation}
for arbitrary analytic function $G$ with
\begin{equation}
  G(x)=\sum_{i=0}^\infty G_ix^i
\end{equation}
and $f$ is
\begin{equation}
  f(x)=xe^{1/x}\operatorname{Ei}(-1/x)
\end{equation}
the Kramers--Kronig transformation of \eqref{eq:im.f.func.2d}, where $\operatorname{Ei}$ is the exponential integral.

\subsection{Iterative fitting}

\subsection{Comparison with other smooth forms}

\section{The three-dimensional Ising model}

The three-dimensional Ising model is easier in some ways, since its hyperbolic critical point lacks stray logarithms.

\begin{equation} \label{eq:free.energy.3d.low}
  f_s(u_t, u_h)
  = |u_t|^{2-\alpha}\mathcal F_{\pm}(u_h|u_t|^{-\beta\delta})
\end{equation}
\begin{equation} \label{eq:free.energy.3d.mid}
  f_s(u_t, u_h)
  = |u_h|^{(2-\alpha)/\beta\delta}\mathcal F_0(u_t|u_h|^{-1/\beta\delta})
\end{equation}
\begin{equation} \label{eq:schofield.3d.free.energy}
  f_s(R, \theta) = R^2\mathcal F(\theta)
\end{equation}
\begin{equation} \label{eq:scaling.function.equivalences.3d}
  \begin{aligned}
    \mathcal F(\theta)
    &=t(\theta)^{2-\alpha}\mathcal F_\pm\left[h(\theta)|t(\theta)|^{-\beta\delta}\right] \\
    &=|h(\theta)|^{(2-\alpha)/\beta\delta}\mathcal F_0\left[t(\theta)|h(\theta)|^{-1/\beta\delta}\right]
  \end{aligned}
\end{equation}
\begin{equation} \label{eq:im.f.func.3d}
  f(x)=\Theta(-x) |x|^{-7/3} e^{-1/|x|^2}
\end{equation}
\begin{equation}
  f(x)=\frac{e^{-1/x^2}}{12}\left[
    \frac4x\Gamma\big(\tfrac23\big)\operatorname{E}_{\frac53}(-x^{-2})
    -\frac1{x^2}\Gamma\big(\tfrac16\big)\operatorname{E}_{\frac76}(-x^{-2})
  \right]
\end{equation}

\section{Outlook}

The successful smooth description of the Ising free energy produced in part by analytically continuing the singular imaginary part of the metastable free energy inspires an extension of this work: a smooth function that captures the universal scaling \emph{through the coexistence line and into the metastable phase}. Indeed, the tools exist to produce this: by writing $t(\theta)=(1-\theta^2)(1-(\theta/\theta_m)^2)$ for some $\theta_m>\theta_c$, the invariant scaling combination

\begin{acknowledgments}
  The authors would like to thank Tom Lubensky, Andrea Liu, and Randy Kamien
  for helpful conversations. The authors would also like to think Jacques Perk
  for pointing us to several insightful studies. JPS thanks Jim Langer for past
  inspiration, guidance, and encouragement. This work was supported by NSF
  grants DMR-1312160 and DMR-1719490.
\end{acknowledgments}

\bibliography{ising_scaling}

\end{document}
