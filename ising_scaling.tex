\documentclass[
  aps,
  prb,
  reprint,
  longbibliography,
  floatfix
]{revtex4-2}

\usepackage[utf8]{inputenc}
\usepackage[T1]{fontenc}
\usepackage{newtxtext, newtxmath}
\usepackage[
  colorlinks=true,
  urlcolor=purple,
  citecolor=purple,
  filecolor=purple,
  linkcolor=purple
]{hyperref}
\usepackage{amsmath}
\usepackage{graphicx}
\usepackage{xcolor}

\begin{document}

\title{Smooth Ising universal scaling functions}

\author{Jaron Kent-Dobias}
\affiliation{Laboratoire de Physique de l'Ecole Normale Supérieure, Paris, France}

\author{James P.~Sethna}
\affiliation{Laboratory of Atomic and Solid State Physics, Cornell University, Ithaca, NY, USA}

\date\today

\begin{abstract}
\end{abstract}

\maketitle

\cite{Campostrini_2000_Critical}


\section{The 2D Ising model}

\subsection{Definition of functions}

\begin{equation} \label{eq:free.energy.2d.low}
  F(u_t, u_h)
  = |u_t|^2\mathcal F_{\pm}(u_h|u_t|^{-\beta\delta})
    +\frac{u_t^2}{8\pi}\log u_t^2
\end{equation}
where the functions $\mathcal F_\pm$ have expansions in nonnegative integer powers of their arguments.
\begin{equation} \label{eq:free.energy.2d.mid}
  F(u_t, u_h)
  = |u_h|^{2/\beta\delta}\mathcal F_0(u_t|u_h|^{-1/\beta\delta})
    +\frac{u_t^2}{8\pi}\log u_h^{2/\beta\delta}
\end{equation}
where the function $\mathcal F_0$ has a convergent expansion in nonnegative integer powers of its argument \footnote{
  To connect with Mangazeev and Fonseca, $\mathcal F_0(x)=\tilde\Phi(-x)=\Phi(-x)+(x^2/8\pi) \log x^2$ and $\mathcal F_\pm(x)=G_{\mathrm{high}/\mathrm{low}}(x)$.
}.

\begin{align}
  \label{eq:schofield.free.energy}
  F(R, \theta) &= R^2\mathcal F(\theta) + t(\theta)^2\frac{R^2}{8\pi}\log R^2 \\
  \label{eq:schofield.temperature}
  u_t(R, \theta) &= Rt(\theta) \\
  \label{eq:schofield.field}
  u_h(R, \theta) &= R^{\beta\delta}h(\theta)
\end{align}
The scaling function $\mathcal F$ can be defined in terms of the more conventional ones above by substituting \eqref{eq:schofield.temperature} and \eqref{eq:schofield.field} into \eqref{eq:free.energy.2d.low} and \eqref{eq:free.energy.2d.mid}, yielding
\begin{equation}
  \begin{aligned}
    &\mathcal F(\theta)
    =t(\theta)^2\mathcal F_\pm\left[h(\theta)|t(\theta)|^{-\beta\delta}\right]
      +\frac{t(\theta)^2}{8\pi}\log t(\theta)^2 \\
    &=|h(\theta)|^{2/\beta\delta}\mathcal F_0\left[t(\theta)|h(\theta)|^{-1/\beta\delta}\right]
      +\frac{t(\theta)^2}{8\pi}\log h(\theta)^{2/\beta\delta}
  \end{aligned}
\end{equation}
We choose the functions $t$ and $h$ so as to ensure that $F$ has an integer power series in \emph{all} regimes. $t$ is an even function of $\theta$ with $t(0)=1$ and $t(1)=0$. $h$ is an odd function with $h(0)=h(\theta_c)=0$ for some $\theta_c>1$.

\begin{align}
  t(\theta)&=1-\theta^2 \\
  h^{(n)}(\theta)&=\left(1-\frac{\theta^2}{\theta_c^2}\right)\sum_{i=0}^nh_i\theta^{2i+1}
\end{align}


\begin{equation}
  f(x)=\Theta(-x) |x| e^{-1/|x|}
\end{equation}
where $\Theta$ is the Heaviside function.

\begin{equation}
  \operatorname{Im}\mathcal F(\theta)=A\left\{f\left[b(\theta_c-\theta)\right]+f\left[b(\theta_c+\theta)\right]\right\}
\end{equation}

\begin{equation}
  \begin{aligned}
    \operatorname{Re}\mathcal F(\theta)
      &=G(\theta^2)-\frac{\theta^2}\pi\int d\vartheta\, \frac{\operatorname{Im}\mathcal F(\vartheta)}{\vartheta^2(\vartheta-\theta)} \\
      &=G(\theta^2)+\frac A\pi\left\{f[b(\theta_c-\theta)]+f[b(\theta_c+\theta)]\right\}
  \end{aligned}
\end{equation}
for arbitrary analytic function $G$ and
\begin{equation}
  f(x)=xe^{1/x}\operatorname{Ei}(-1/x)
\end{equation}

\section{The 3D Ising model}

\section{Outlook}

\begin{acknowledgments}
  The authors would like to thank Tom Lubensky, Andrea Liu, and Randy Kamien
  for helpful conversations. The authors would also like to think Jacques Perk
  for pointing us to several insightful studies. JPS thanks Jim Langer for past
  inspiration, guidance, and encouragement. This work was supported by NSF
  grants DMR-1312160 and DMR-1719490.
\end{acknowledgments}

\bibliography{ising_scaling}

\end{document}
