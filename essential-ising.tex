%  Ising model abrupt transition.
%
%  Created by Jaron Kent-Dobias on Thu Apr 20 12:50:56 EDT 2017.
%  Copyright (c) 2017 Jaron Kent-Dobias. All rights reserved.
%
\documentclass[fleqn]{article}

\usepackage[utf8]{inputenc}
\usepackage[T1]{fontenc}
\usepackage{amsmath,amssymb,latexsym,concmath,mathtools,xifthen,mfpic}

\mathtoolsset{showonlyrefs=true}

\title{Essential Singularity in the Ising Abrupt Transition}
\author{Jaron Kent-Dobias}

\date{April 20, 2017}

\begin{document}

\def\[{\begin{equation}}
\def\]{\end{equation}}

\def\im{\mathop{\mathrm{Im}}\nolimits}
\def\dd{\mathrm d}
\def\O{\mathcal O}
\def\ei{\mathop{\mathrm{Ei}}\nolimits}
\def\b{\mathrm b}

\newcommand\pd[3][]{
  \ifthenelse{\isempty{#1}}
    {\def\tmp{}}
    {\def\tmp{^#1}}
  \frac{\partial\tmp#2}{\partial#3\tmp}
}

\maketitle

\begin{abstract}
\end{abstract}

It's long been known that the decay rate $\Gamma$ of metastable states in
statistical mechanics is often related to the metastable free energy $F$ by
\cite{langer.1967.condensation,langer.1969.metastable,gaveau.1989.analytic}
\[
  \Gamma\propto\im F
\]
What exactly is meant by `metastable free energy' is important to establish,
since formally the free energy relies on the existence of an equilibrium
state. Here one can imagine either analytic continuation of the free energy
through an abrupt phase transition, or restriction of the partition function
trace to states in the vicinity of the local free energy minimum that
characterizes the metastable state. In any case, the free energy develops a
nonzero imaginary part in the metastable region. Heuristically, this can be
thought of as similar to what happens in quantum mechanics with a non-unitary
Hamiltonian: the imaginary part describes loss of probability in the system
that corresponds to decay. 

One can estimate the scaling of the decay rate of the {\sc 2d} Ising model
using ideas from nucleation theory. In this framework, the metastable state
decays when a sufficiently large domain in the stable state forms to grow
stably to fill out the whole system. The free energy of a domain of $N$ spins
causes a free energy change
\[
  \Delta F=\Sigma N^\sigma-MHN
\]
where $\Sigma$ is the surface tension and $1-\frac1d\leq\sigma<1$. This is
maximized by
\[
  N_c=\bigg(\frac{MH}{\sigma\Sigma}\bigg)^{-1/(\sigma-1)}
\]
which corresponds to a free energy change
\[
  \Delta F_c\sim\bigg(\frac\Sigma{(MH)^\sigma}\bigg)^{1/(1-\sigma)}
\]
The rate of formation is proportional to the Boltzmann factor,
\[
  \Gamma\sim e^{-\beta \Delta
  F_c}=e^{-\beta(\Sigma/(MH)^\sigma)^{1/(1-\sigma)}}
\]
For domains whose boundary is minimal, $\sigma=1-\frac1d$ and this becomes
\[
  \Gamma\sim e^{-\beta(\Sigma/(MH)^\sigma)^{d-1}}
\]
Since $\Sigma\sim t^\mu\mathcal S(ht^{-\beta\delta})$ with $\mu=-\nu+\gamma+2\beta$
\cite{widom.1981.interface} and $M\sim t^\beta\mathcal M(ht^{-\beta\delta})$
with $\mathcal S(0)=\O(1)$ and $\mathcal M(0)=\O(1)$,
\[
  \Gamma\sim e^{-1/\mathcal G(ht^{-\beta\delta})^{d-1}}
\]
with $\mathcal G(X)=\O(X)$. This establishes the form of $\im F$
besides the prefactor. Results from field theory predict that, for small $H$
and $1<d<5$, $d\neq 3$,
\[
  \im F\simeq\bigg(\frac h{t^\Delta}\bigg)^{-(d-3)d/2}(g^*)^{-d(d-1)/4}
  \exp\bigg[-B\bigg(\frac h{|t|^\Delta}\bigg)^{-(d-1)}(g^*)^{-(d+1)/2}\bigg]
\]
\[
  \im F\simeq\bigg(\frac
  h{t^\Delta}\bigg)^{-7/3}(g^*)^{-8/3}\exp\bigg[-B\bigg(\frac
  h{t^\Delta}\bigg)^{-2}(g^*)^{-2}\bigg]
\]
with $\Delta=3-\frac\epsilon2$, $g^*=2\pi^2\frac\epsilon{n+8}$
\cite{houghton.1980.metastable,gunther.1980.goldstone}. This is consistent
with our form above. We therefore predict that
\[
  \im F=t^{2-\alpha}\mathcal F(ht^{-\beta\delta})^{-(d-3)d/2}e^{-1/\mathcal
    G(ht^{-\beta\delta})^{d-1}}
\]
In {\sc 2d} we have
\[
  \im F=t^2\mathcal F(ht^{-\Delta})e^{-1/\mathcal G(ht^{-\Delta})}
\]
with $\Delta=\beta\delta=\frac{15}8$. In terms of $X=ht^{-\Delta}$, this is
\[
  \im F=t^2\mathcal F(X)e^{-1/\mathcal G(X)}\simeq At^2|X|e^{-1/B|X|}
\]

\cite{langer.1967.condensation}

\[
  F(X)=\frac1\pi\int_{-\infty}^\infty\frac{\im F(X')}{X'-X}\,\dd X'
  =\frac{At^2}\pi\int_{-\infty}^0\frac{|X'|e^{-1/B|X'|}}{X'-X}\,\dd
  X'
  =-\frac{At^2}\pi\int_0^\infty\frac{X'e^{-1/BX'}}{X'+X}\,\dd
  X'
\]
since $\im F=0$ for $X>0$. $\pd{}h=\pd Xh\pd{}X=t^{-\Delta}\pd{}X$.
Unfortunately this integral doesn't converge, and it seems we cannot evaluate
this result at the level of truncation we've chosen. However, 

\[
  F(H)=At^{2-\alpha}\sum_{n=0}^\infty f_nX^n
\]
\[
  f_n=\frac1\pi\int_{-\infty}^0\frac{\im F(X)}{X^{n+1}}\,\dd X
  =\frac{(-1)^{n+1}}\pi\int_0^{\infty}\frac{Xe^{-1/BX}}{X^{n+1}}\,\dd X
  =\frac1\pi(-1)^{n+1}B^{n-1}\Gamma(n-1)
\]
for $n>1$. 

\begin{align}
  \chi
  &=\pd[2]Fh
  =t^{-2\Delta}\pd[2]FX
  =-\frac{2}\pi At^{2-2\Delta}\int_0^\infty\frac{X'e^{-1/BX'}}{(X+X')^3}\,\dd
  X'\\
  &=\frac2\pi
  \frac{ABt^{-\gamma}}{(BX)^3}\big[BX(1-BX)+e^{1/BX}\ei(-1/BX)\big]
\end{align}

\[
  \lim_{X\to0}\chi=-\frac4\pi ABt^{-\gamma}
\]

\[
  \beta^{-1}\chi=C_{0\pm}|t|^{-7/4}+C_{1\pm}|t|^{-3/4}+\O(1)
\]
$C_{0-}=0.025\,536\,971\,9$ $C_{1-}=-0.001\,989\,410\,7$
\cite{barouch.1973.susceptibility}

CORRECTIONS TO SCALING, $u_t$ and $u_h$ instead of $t$ and $h$.

\begin{align}
  u_h
  &=h[1+c_ht+dht^2+e_hh^2+f_ht^3+\O(t^4,th^2)]\\
  u_t
  &=t+b_th^2+c_t^2+d_t^3+e_tth^2+f_tt^4+\O(t^5,t^2h^2,h^4)
\end{align}
\begin{align}
  c_h=\frac{\beta_c}{\sqrt2}
  &&
  d_h=\frac{23\beta_c^2}{16}
  &&
  f_h=\frac{191\beta_c^3}{48\sqrt2}\\
  c_t=\frac{\beta_c}{\sqrt2}
  &&
  d_t=\frac{7\beta_c^2}6
  &&
  f_t=\frac{17\beta_c^3}{6\sqrt2}\\
  e_t=b_t\beta_c\sqrt2
  &&
  b_t=-\frac{E_0\pi}{16\beta_c^2}
\end{align}
$E_0=0.040\,325\,5003$ $e_h=-0.007\,27(15)$
\[
  F(t,h)-F(t,0)=\sum_{n=1}^\infty\frac1{(2n)!}\chi_{2n}(t)h^{2n}
\]
\[
  \chi(t,h)=\pd[2]Fh=\chi_2(t)+\sum_{n=1}^\infty\frac1{(2n)!}\chi_{2(n+1)}h^{2n}
\]

\begin{align}
  \chi
  &=\pd[2]Fh
  =\pd[2]{F_\b}h
  +\frac d{y_t}\bigg(\frac d{y_t}-1\bigg)|u_t|^{d/y_t-2}\bigg(\pd{u_t}h\bigg)^2
\end{align}

% GNUPLOT: LaTeX picture
\setlength{\unitlength}{0.240900pt}
\ifx\plotpoint\undefined\newsavebox{\plotpoint}\fi
\sbox{\plotpoint}{\rule[-0.200pt]{0.400pt}{0.400pt}}%
\begin{picture}(1500,900)(0,0)
\sbox{\plotpoint}{\rule[-0.200pt]{0.400pt}{0.400pt}}%
\put(151.0,130.0){\rule[-0.200pt]{4.818pt}{0.400pt}}
\put(131,130){\makebox(0,0)[r]{$0$}}
\put(1419.0,130.0){\rule[-0.200pt]{4.818pt}{0.400pt}}
\put(151.0,251.0){\rule[-0.200pt]{4.818pt}{0.400pt}}
\put(131,251){\makebox(0,0)[r]{$0.1$}}
\put(1419.0,251.0){\rule[-0.200pt]{4.818pt}{0.400pt}}
\put(151.0,373.0){\rule[-0.200pt]{4.818pt}{0.400pt}}
\put(131,373){\makebox(0,0)[r]{$0.2$}}
\put(1419.0,373.0){\rule[-0.200pt]{4.818pt}{0.400pt}}
\put(151.0,494.0){\rule[-0.200pt]{4.818pt}{0.400pt}}
\put(131,494){\makebox(0,0)[r]{$0.3$}}
\put(1419.0,494.0){\rule[-0.200pt]{4.818pt}{0.400pt}}
\put(151.0,616.0){\rule[-0.200pt]{4.818pt}{0.400pt}}
\put(131,616){\makebox(0,0)[r]{$0.4$}}
\put(1419.0,616.0){\rule[-0.200pt]{4.818pt}{0.400pt}}
\put(151.0,737.0){\rule[-0.200pt]{4.818pt}{0.400pt}}
\put(131,737){\makebox(0,0)[r]{$0.5$}}
\put(1419.0,737.0){\rule[-0.200pt]{4.818pt}{0.400pt}}
\put(151.0,859.0){\rule[-0.200pt]{4.818pt}{0.400pt}}
\put(131,859){\makebox(0,0)[r]{$0.6$}}
\put(1419.0,859.0){\rule[-0.200pt]{4.818pt}{0.400pt}}
\put(151.0,131.0){\rule[-0.200pt]{0.400pt}{4.818pt}}
\put(151,90){\makebox(0,0){$0$}}
\put(151.0,839.0){\rule[-0.200pt]{0.400pt}{4.818pt}}
\put(280.0,131.0){\rule[-0.200pt]{0.400pt}{4.818pt}}
\put(280,90){\makebox(0,0){$0.5$}}
\put(280.0,839.0){\rule[-0.200pt]{0.400pt}{4.818pt}}
\put(408.0,131.0){\rule[-0.200pt]{0.400pt}{4.818pt}}
\put(408,90){\makebox(0,0){$1$}}
\put(408.0,839.0){\rule[-0.200pt]{0.400pt}{4.818pt}}
\put(537.0,131.0){\rule[-0.200pt]{0.400pt}{4.818pt}}
\put(537,90){\makebox(0,0){$1.5$}}
\put(537.0,839.0){\rule[-0.200pt]{0.400pt}{4.818pt}}
\put(666.0,131.0){\rule[-0.200pt]{0.400pt}{4.818pt}}
\put(666,90){\makebox(0,0){$2$}}
\put(666.0,839.0){\rule[-0.200pt]{0.400pt}{4.818pt}}
\put(795.0,131.0){\rule[-0.200pt]{0.400pt}{4.818pt}}
\put(795,90){\makebox(0,0){$2.5$}}
\put(795.0,839.0){\rule[-0.200pt]{0.400pt}{4.818pt}}
\put(924.0,131.0){\rule[-0.200pt]{0.400pt}{4.818pt}}
\put(924,90){\makebox(0,0){$3$}}
\put(924.0,839.0){\rule[-0.200pt]{0.400pt}{4.818pt}}
\put(1053.0,131.0){\rule[-0.200pt]{0.400pt}{4.818pt}}
\put(1053,90){\makebox(0,0){$3.5$}}
\put(1053.0,839.0){\rule[-0.200pt]{0.400pt}{4.818pt}}
\put(1181.0,131.0){\rule[-0.200pt]{0.400pt}{4.818pt}}
\put(1181,90){\makebox(0,0){$4$}}
\put(1181.0,839.0){\rule[-0.200pt]{0.400pt}{4.818pt}}
\put(1310.0,131.0){\rule[-0.200pt]{0.400pt}{4.818pt}}
\put(1310,90){\makebox(0,0){$4.5$}}
\put(1310.0,839.0){\rule[-0.200pt]{0.400pt}{4.818pt}}
\put(1439.0,131.0){\rule[-0.200pt]{0.400pt}{4.818pt}}
\put(1439,90){\makebox(0,0){$5$}}
\put(1439.0,839.0){\rule[-0.200pt]{0.400pt}{4.818pt}}
\put(151.0,131.0){\rule[-0.200pt]{0.400pt}{175.375pt}}
\put(151.0,131.0){\rule[-0.200pt]{310.279pt}{0.400pt}}
\put(1439.0,131.0){\rule[-0.200pt]{0.400pt}{175.375pt}}
\put(151.0,859.0){\rule[-0.200pt]{310.279pt}{0.400pt}}
\put(30,495){\rotatebox{-270}{\makebox(0,0){$\chi B/At^{-\gamma}$}}
}\put(795,29){\makebox(0,0){$h/Bt^{\beta\delta}$}}
\multiput(177.58,724.43)(0.493,-2.201){23}{\rule{0.119pt}{1.823pt}}
\multiput(176.17,728.22)(13.000,-52.216){2}{\rule{0.400pt}{0.912pt}}
\multiput(190.58,669.84)(0.493,-1.765){23}{\rule{0.119pt}{1.485pt}}
\multiput(189.17,672.92)(13.000,-41.919){2}{\rule{0.400pt}{0.742pt}}
\multiput(203.58,625.86)(0.493,-1.448){23}{\rule{0.119pt}{1.238pt}}
\multiput(202.17,628.43)(13.000,-34.430){2}{\rule{0.400pt}{0.619pt}}
\multiput(216.58,589.50)(0.493,-1.250){23}{\rule{0.119pt}{1.085pt}}
\multiput(215.17,591.75)(13.000,-29.749){2}{\rule{0.400pt}{0.542pt}}
\multiput(229.58,558.14)(0.493,-1.052){23}{\rule{0.119pt}{0.931pt}}
\multiput(228.17,560.07)(13.000,-25.068){2}{\rule{0.400pt}{0.465pt}}
\multiput(242.58,531.65)(0.493,-0.893){23}{\rule{0.119pt}{0.808pt}}
\multiput(241.17,533.32)(13.000,-21.324){2}{\rule{0.400pt}{0.404pt}}
\multiput(255.58,508.90)(0.493,-0.814){23}{\rule{0.119pt}{0.746pt}}
\multiput(254.17,510.45)(13.000,-19.451){2}{\rule{0.400pt}{0.373pt}}
\multiput(268.58,488.16)(0.493,-0.734){23}{\rule{0.119pt}{0.685pt}}
\multiput(267.17,489.58)(13.000,-17.579){2}{\rule{0.400pt}{0.342pt}}
\multiput(281.58,469.54)(0.493,-0.616){23}{\rule{0.119pt}{0.592pt}}
\multiput(280.17,470.77)(13.000,-14.771){2}{\rule{0.400pt}{0.296pt}}
\multiput(294.58,453.67)(0.493,-0.576){23}{\rule{0.119pt}{0.562pt}}
\multiput(293.17,454.83)(13.000,-13.834){2}{\rule{0.400pt}{0.281pt}}
\multiput(307.58,438.80)(0.493,-0.536){23}{\rule{0.119pt}{0.531pt}}
\multiput(306.17,439.90)(13.000,-12.898){2}{\rule{0.400pt}{0.265pt}}
\multiput(320.00,425.92)(0.539,-0.492){21}{\rule{0.533pt}{0.119pt}}
\multiput(320.00,426.17)(11.893,-12.000){2}{\rule{0.267pt}{0.400pt}}
\multiput(333.00,413.92)(0.590,-0.492){19}{\rule{0.573pt}{0.118pt}}
\multiput(333.00,414.17)(11.811,-11.000){2}{\rule{0.286pt}{0.400pt}}
\multiput(346.00,402.92)(0.590,-0.492){19}{\rule{0.573pt}{0.118pt}}
\multiput(346.00,403.17)(11.811,-11.000){2}{\rule{0.286pt}{0.400pt}}
\multiput(359.00,391.93)(0.728,-0.489){15}{\rule{0.678pt}{0.118pt}}
\multiput(359.00,392.17)(11.593,-9.000){2}{\rule{0.339pt}{0.400pt}}
\multiput(372.00,382.93)(0.728,-0.489){15}{\rule{0.678pt}{0.118pt}}
\multiput(372.00,383.17)(11.593,-9.000){2}{\rule{0.339pt}{0.400pt}}
\multiput(385.00,373.93)(0.824,-0.488){13}{\rule{0.750pt}{0.117pt}}
\multiput(385.00,374.17)(11.443,-8.000){2}{\rule{0.375pt}{0.400pt}}
\multiput(398.00,365.93)(0.824,-0.488){13}{\rule{0.750pt}{0.117pt}}
\multiput(398.00,366.17)(11.443,-8.000){2}{\rule{0.375pt}{0.400pt}}
\multiput(411.00,357.93)(0.950,-0.485){11}{\rule{0.843pt}{0.117pt}}
\multiput(411.00,358.17)(11.251,-7.000){2}{\rule{0.421pt}{0.400pt}}
\multiput(424.00,350.93)(0.950,-0.485){11}{\rule{0.843pt}{0.117pt}}
\multiput(424.00,351.17)(11.251,-7.000){2}{\rule{0.421pt}{0.400pt}}
\multiput(437.00,343.93)(1.123,-0.482){9}{\rule{0.967pt}{0.116pt}}
\multiput(437.00,344.17)(10.994,-6.000){2}{\rule{0.483pt}{0.400pt}}
\multiput(450.00,337.93)(1.123,-0.482){9}{\rule{0.967pt}{0.116pt}}
\multiput(450.00,338.17)(10.994,-6.000){2}{\rule{0.483pt}{0.400pt}}
\multiput(463.00,331.93)(1.123,-0.482){9}{\rule{0.967pt}{0.116pt}}
\multiput(463.00,332.17)(10.994,-6.000){2}{\rule{0.483pt}{0.400pt}}
\multiput(476.00,325.93)(1.378,-0.477){7}{\rule{1.140pt}{0.115pt}}
\multiput(476.00,326.17)(10.634,-5.000){2}{\rule{0.570pt}{0.400pt}}
\multiput(489.00,320.93)(1.378,-0.477){7}{\rule{1.140pt}{0.115pt}}
\multiput(489.00,321.17)(10.634,-5.000){2}{\rule{0.570pt}{0.400pt}}
\multiput(502.00,315.93)(1.378,-0.477){7}{\rule{1.140pt}{0.115pt}}
\multiput(502.00,316.17)(10.634,-5.000){2}{\rule{0.570pt}{0.400pt}}
\multiput(515.00,310.94)(1.797,-0.468){5}{\rule{1.400pt}{0.113pt}}
\multiput(515.00,311.17)(10.094,-4.000){2}{\rule{0.700pt}{0.400pt}}
\multiput(528.00,306.93)(1.378,-0.477){7}{\rule{1.140pt}{0.115pt}}
\multiput(528.00,307.17)(10.634,-5.000){2}{\rule{0.570pt}{0.400pt}}
\multiput(541.00,301.94)(1.797,-0.468){5}{\rule{1.400pt}{0.113pt}}
\multiput(541.00,302.17)(10.094,-4.000){2}{\rule{0.700pt}{0.400pt}}
\multiput(554.00,297.94)(1.797,-0.468){5}{\rule{1.400pt}{0.113pt}}
\multiput(554.00,298.17)(10.094,-4.000){2}{\rule{0.700pt}{0.400pt}}
\multiput(567.00,293.95)(2.695,-0.447){3}{\rule{1.833pt}{0.108pt}}
\multiput(567.00,294.17)(9.195,-3.000){2}{\rule{0.917pt}{0.400pt}}
\multiput(580.00,290.94)(1.797,-0.468){5}{\rule{1.400pt}{0.113pt}}
\multiput(580.00,291.17)(10.094,-4.000){2}{\rule{0.700pt}{0.400pt}}
\multiput(593.00,286.95)(2.695,-0.447){3}{\rule{1.833pt}{0.108pt}}
\multiput(593.00,287.17)(9.195,-3.000){2}{\rule{0.917pt}{0.400pt}}
\multiput(606.00,283.95)(2.695,-0.447){3}{\rule{1.833pt}{0.108pt}}
\multiput(606.00,284.17)(9.195,-3.000){2}{\rule{0.917pt}{0.400pt}}
\multiput(619.00,280.95)(2.695,-0.447){3}{\rule{1.833pt}{0.108pt}}
\multiput(619.00,281.17)(9.195,-3.000){2}{\rule{0.917pt}{0.400pt}}
\multiput(632.00,277.95)(2.695,-0.447){3}{\rule{1.833pt}{0.108pt}}
\multiput(632.00,278.17)(9.195,-3.000){2}{\rule{0.917pt}{0.400pt}}
\multiput(645.00,274.95)(2.695,-0.447){3}{\rule{1.833pt}{0.108pt}}
\multiput(645.00,275.17)(9.195,-3.000){2}{\rule{0.917pt}{0.400pt}}
\multiput(658.00,271.95)(2.695,-0.447){3}{\rule{1.833pt}{0.108pt}}
\multiput(658.00,272.17)(9.195,-3.000){2}{\rule{0.917pt}{0.400pt}}
\multiput(671.00,268.95)(2.695,-0.447){3}{\rule{1.833pt}{0.108pt}}
\multiput(671.00,269.17)(9.195,-3.000){2}{\rule{0.917pt}{0.400pt}}
\put(684,265.17){\rule{2.700pt}{0.400pt}}
\multiput(684.00,266.17)(7.396,-2.000){2}{\rule{1.350pt}{0.400pt}}
\multiput(697.00,263.95)(2.695,-0.447){3}{\rule{1.833pt}{0.108pt}}
\multiput(697.00,264.17)(9.195,-3.000){2}{\rule{0.917pt}{0.400pt}}
\put(710,260.17){\rule{2.700pt}{0.400pt}}
\multiput(710.00,261.17)(7.396,-2.000){2}{\rule{1.350pt}{0.400pt}}
\put(723,258.17){\rule{2.700pt}{0.400pt}}
\multiput(723.00,259.17)(7.396,-2.000){2}{\rule{1.350pt}{0.400pt}}
\put(736,256.17){\rule{2.700pt}{0.400pt}}
\multiput(736.00,257.17)(7.396,-2.000){2}{\rule{1.350pt}{0.400pt}}
\put(749,254.17){\rule{2.700pt}{0.400pt}}
\multiput(749.00,255.17)(7.396,-2.000){2}{\rule{1.350pt}{0.400pt}}
\multiput(762.00,252.95)(2.695,-0.447){3}{\rule{1.833pt}{0.108pt}}
\multiput(762.00,253.17)(9.195,-3.000){2}{\rule{0.917pt}{0.400pt}}
\put(775,249.67){\rule{3.132pt}{0.400pt}}
\multiput(775.00,250.17)(6.500,-1.000){2}{\rule{1.566pt}{0.400pt}}
\put(788,248.17){\rule{2.900pt}{0.400pt}}
\multiput(788.00,249.17)(7.981,-2.000){2}{\rule{1.450pt}{0.400pt}}
\put(802,246.17){\rule{2.700pt}{0.400pt}}
\multiput(802.00,247.17)(7.396,-2.000){2}{\rule{1.350pt}{0.400pt}}
\put(815,244.17){\rule{2.700pt}{0.400pt}}
\multiput(815.00,245.17)(7.396,-2.000){2}{\rule{1.350pt}{0.400pt}}
\put(828,242.17){\rule{2.700pt}{0.400pt}}
\multiput(828.00,243.17)(7.396,-2.000){2}{\rule{1.350pt}{0.400pt}}
\put(841,240.67){\rule{3.132pt}{0.400pt}}
\multiput(841.00,241.17)(6.500,-1.000){2}{\rule{1.566pt}{0.400pt}}
\put(854,239.17){\rule{2.700pt}{0.400pt}}
\multiput(854.00,240.17)(7.396,-2.000){2}{\rule{1.350pt}{0.400pt}}
\put(867,237.17){\rule{2.700pt}{0.400pt}}
\multiput(867.00,238.17)(7.396,-2.000){2}{\rule{1.350pt}{0.400pt}}
\put(880,235.67){\rule{3.132pt}{0.400pt}}
\multiput(880.00,236.17)(6.500,-1.000){2}{\rule{1.566pt}{0.400pt}}
\put(893,234.17){\rule{2.700pt}{0.400pt}}
\multiput(893.00,235.17)(7.396,-2.000){2}{\rule{1.350pt}{0.400pt}}
\put(906,232.67){\rule{3.132pt}{0.400pt}}
\multiput(906.00,233.17)(6.500,-1.000){2}{\rule{1.566pt}{0.400pt}}
\put(919,231.17){\rule{2.700pt}{0.400pt}}
\multiput(919.00,232.17)(7.396,-2.000){2}{\rule{1.350pt}{0.400pt}}
\put(932,229.67){\rule{3.132pt}{0.400pt}}
\multiput(932.00,230.17)(6.500,-1.000){2}{\rule{1.566pt}{0.400pt}}
\put(945,228.67){\rule{3.132pt}{0.400pt}}
\multiput(945.00,229.17)(6.500,-1.000){2}{\rule{1.566pt}{0.400pt}}
\put(958,227.17){\rule{2.700pt}{0.400pt}}
\multiput(958.00,228.17)(7.396,-2.000){2}{\rule{1.350pt}{0.400pt}}
\put(971,225.67){\rule{3.132pt}{0.400pt}}
\multiput(971.00,226.17)(6.500,-1.000){2}{\rule{1.566pt}{0.400pt}}
\put(984,224.67){\rule{3.132pt}{0.400pt}}
\multiput(984.00,225.17)(6.500,-1.000){2}{\rule{1.566pt}{0.400pt}}
\put(997,223.67){\rule{3.132pt}{0.400pt}}
\multiput(997.00,224.17)(6.500,-1.000){2}{\rule{1.566pt}{0.400pt}}
\put(1010,222.67){\rule{3.132pt}{0.400pt}}
\multiput(1010.00,223.17)(6.500,-1.000){2}{\rule{1.566pt}{0.400pt}}
\put(1023,221.17){\rule{2.700pt}{0.400pt}}
\multiput(1023.00,222.17)(7.396,-2.000){2}{\rule{1.350pt}{0.400pt}}
\put(1036,219.67){\rule{3.132pt}{0.400pt}}
\multiput(1036.00,220.17)(6.500,-1.000){2}{\rule{1.566pt}{0.400pt}}
\put(1049,218.67){\rule{3.132pt}{0.400pt}}
\multiput(1049.00,219.17)(6.500,-1.000){2}{\rule{1.566pt}{0.400pt}}
\put(1062,217.67){\rule{3.132pt}{0.400pt}}
\multiput(1062.00,218.17)(6.500,-1.000){2}{\rule{1.566pt}{0.400pt}}
\put(1075,216.67){\rule{3.132pt}{0.400pt}}
\multiput(1075.00,217.17)(6.500,-1.000){2}{\rule{1.566pt}{0.400pt}}
\put(1088,215.67){\rule{3.132pt}{0.400pt}}
\multiput(1088.00,216.17)(6.500,-1.000){2}{\rule{1.566pt}{0.400pt}}
\put(1101,214.67){\rule{3.132pt}{0.400pt}}
\multiput(1101.00,215.17)(6.500,-1.000){2}{\rule{1.566pt}{0.400pt}}
\put(1114,213.67){\rule{3.132pt}{0.400pt}}
\multiput(1114.00,214.17)(6.500,-1.000){2}{\rule{1.566pt}{0.400pt}}
\put(1127,212.67){\rule{3.132pt}{0.400pt}}
\multiput(1127.00,213.17)(6.500,-1.000){2}{\rule{1.566pt}{0.400pt}}
\put(1140,211.67){\rule{3.132pt}{0.400pt}}
\multiput(1140.00,212.17)(6.500,-1.000){2}{\rule{1.566pt}{0.400pt}}
\put(1153,210.67){\rule{3.132pt}{0.400pt}}
\multiput(1153.00,211.17)(6.500,-1.000){2}{\rule{1.566pt}{0.400pt}}
\put(177.0,732.0){\rule[-0.200pt]{0.400pt}{30.594pt}}
\put(1179,209.67){\rule{3.132pt}{0.400pt}}
\multiput(1179.00,210.17)(6.500,-1.000){2}{\rule{1.566pt}{0.400pt}}
\put(1192,208.67){\rule{3.132pt}{0.400pt}}
\multiput(1192.00,209.17)(6.500,-1.000){2}{\rule{1.566pt}{0.400pt}}
\put(1205,207.67){\rule{3.132pt}{0.400pt}}
\multiput(1205.00,208.17)(6.500,-1.000){2}{\rule{1.566pt}{0.400pt}}
\put(1218,206.67){\rule{3.132pt}{0.400pt}}
\multiput(1218.00,207.17)(6.500,-1.000){2}{\rule{1.566pt}{0.400pt}}
\put(1231,205.67){\rule{3.132pt}{0.400pt}}
\multiput(1231.00,206.17)(6.500,-1.000){2}{\rule{1.566pt}{0.400pt}}
\put(1166.0,211.0){\rule[-0.200pt]{3.132pt}{0.400pt}}
\put(1257,204.67){\rule{3.132pt}{0.400pt}}
\multiput(1257.00,205.17)(6.500,-1.000){2}{\rule{1.566pt}{0.400pt}}
\put(1270,203.67){\rule{3.132pt}{0.400pt}}
\multiput(1270.00,204.17)(6.500,-1.000){2}{\rule{1.566pt}{0.400pt}}
\put(1283,202.67){\rule{3.132pt}{0.400pt}}
\multiput(1283.00,203.17)(6.500,-1.000){2}{\rule{1.566pt}{0.400pt}}
\put(1244.0,206.0){\rule[-0.200pt]{3.132pt}{0.400pt}}
\put(1309,201.67){\rule{3.132pt}{0.400pt}}
\multiput(1309.00,202.17)(6.500,-1.000){2}{\rule{1.566pt}{0.400pt}}
\put(1322,200.67){\rule{3.132pt}{0.400pt}}
\multiput(1322.00,201.17)(6.500,-1.000){2}{\rule{1.566pt}{0.400pt}}
\put(1296.0,203.0){\rule[-0.200pt]{3.132pt}{0.400pt}}
\put(1348,199.67){\rule{3.132pt}{0.400pt}}
\multiput(1348.00,200.17)(6.500,-1.000){2}{\rule{1.566pt}{0.400pt}}
\put(1361,198.67){\rule{3.132pt}{0.400pt}}
\multiput(1361.00,199.17)(6.500,-1.000){2}{\rule{1.566pt}{0.400pt}}
\put(1335.0,201.0){\rule[-0.200pt]{3.132pt}{0.400pt}}
\put(1387,197.67){\rule{3.132pt}{0.400pt}}
\multiput(1387.00,198.17)(6.500,-1.000){2}{\rule{1.566pt}{0.400pt}}
\put(1400,196.67){\rule{3.132pt}{0.400pt}}
\multiput(1400.00,197.17)(6.500,-1.000){2}{\rule{1.566pt}{0.400pt}}
\put(1374.0,199.0){\rule[-0.200pt]{3.132pt}{0.400pt}}
\put(1426,195.67){\rule{3.132pt}{0.400pt}}
\multiput(1426.00,196.17)(6.500,-1.000){2}{\rule{1.566pt}{0.400pt}}
\put(1413.0,197.0){\rule[-0.200pt]{3.132pt}{0.400pt}}
\put(151.0,131.0){\rule[-0.200pt]{0.400pt}{175.375pt}}
\put(151.0,131.0){\rule[-0.200pt]{310.279pt}{0.400pt}}
\put(1439.0,131.0){\rule[-0.200pt]{0.400pt}{175.375pt}}
\put(151.0,859.0){\rule[-0.200pt]{310.279pt}{0.400pt}}
\end{picture}


\bibliographystyle{plain}
\bibliography{essential-ising}

\end{document}

