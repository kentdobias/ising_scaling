
%
%  Created by Jaron Kent-Dobias on Thu Apr 20 12:50:56 EDT 2017.
%  Copyright (c) 2017 Jaron Kent-Dobias. All rights reserved.
%
\documentclass[aps,prl,reprint]{revtex4-1}

\usepackage[utf8]{inputenc}
\usepackage{amsmath,amssymb,latexsym,mathtools,xifthen}

% uncomment to label only equations that are referenced in the text
%\mathtoolsset{showonlyrefs=true}

% I want labels but don't want to type out ``equation''
\def\[{\begin{equation}}
\def\]{\end{equation}}

% math not built-in
\def\arcsinh{\mathop{\mathrm{arcsinh}}\nolimits}
\def\arccosh{\mathop{\mathrm{arccosh}}\nolimits}
\def\ei{\mathop{\mathrm{Ei}}\nolimits} % exponential integral Ei
\def\re{\mathop{\mathrm{Re}}\nolimits}
\def\im{\mathop{\mathrm{Im}}\nolimits}
\def\sgn{\mathop{\mathrm{sgn}}\nolimits}
\def\dd{d} % differential
\def\O{O}          % big O
\def\o{o}          % little O

% subscript for ``critical'' values, e.g., T_\c
\def\c{\mathrm c}

% scaling functions
\def\fM{\mathcal M}  % magnetization
\def\fX{\mathcal Y}  % susceptibility
\def\fF{\mathcal F}  % free energy
\def\fiF{\mathcal H} % imaginary free energy
\def\fS{\mathcal S}  % surface tension
\def\fG{\mathcal G}  % exponential factor

% lattice types
\def\sq{\mathrm{sq}}
\def\tri{\mathrm{tri}}
\def\hex{\mathrm{hex}}

% dimensions
\def\dim{d}
\def\twodee{\textsc{2d} }
\def\threedee{\textsc{3d} }
\def\fourdee{\textsc{4d} }

% fancy partial derivative
\newcommand\pd[3][]{
  \ifthenelse{\isempty{#1}}
    {\def\tmp{}}
    {\def\tmp{^#1}}
  \frac{\partial\tmp#2}{\partial#3\tmp}
}

% used to reformat display math to fit in two-column ``reprint'' mode
\makeatletter
\newif\ifreprint
\@ifclasswith{revtex4-1}{reprint}{\reprinttrue}{\reprintfalse}
\makeatother

\begin{document}

\title{Essential Singularities in the Ising Universal Scaling Functions}
\author{Jaron Kent-Dobias}
\author{James P.~Sethna}
\affiliation{Cornell University}

\date\today

\begin{abstract}
  Renormalization group ideas and results from critical droplet theory are
  used to construct a scaling ansatz for the imaginary component of the free
  energy of an Ising model in its metastable state close to the critical
  point. The analytic properties of the free energy are used to determine
  asymptotic scaling functions for the free energy in the vicinity of the
  critical point and the abrupt transition line. These functions have
  essential singularities at zero field. Analogous forms for the magnetization
  and susceptibility in two-dimensions are fit to numeric data and show good
  agreement.
\end{abstract}

\maketitle

The Ising model is the canonical example of a system with a continuous phase
transition, and the study of its singular properties marked the first success
of the renormalization group (\textsc{rg}) method in statistical physics
\cite{wilson.1971.renormalization}. This status makes sense: it's a simple
model whose phase transition admits \textsc{rg} methods in a straightforward way,
and has exact solutions in certain dimensions and for certain parameter
restrictions. However, in one respect the Ising critical point is not simply a
continuous transition: it ends the line of abrupt phase transitions at zero
field below the critical temperature. Though typically neglected in \textsc{rg}
scaling analyses of the critical point, we demonstrate that there are
numerically measurable contributions to scaling due to the abrupt transition
line that cannot be accounted for by analytic changes of control or
thermodynamic variables.

\textsc{Rg} analysis predicts that the singular part of the free energy per
site $F$ as a function of reduced temperature $t=1-T_\c/T$ and field $h=H/T$ in
the vicinity of the critical point takes the scaling form
$F(t,h)=|t|^{2-\alpha}\fF(h|t|^{-\Delta})$ \footnote{Technically
we should write $\fF_{\pm}$ to indicate that the universal scaling
function takes a different form for $t<0$ and $t>0$, but we will restrict
ourselves entirely to $t<0$ and hence $\fF_-$ for the purposes of this
paper.}, where $\Delta=\beta\delta$ and $t$, $h$ are analytic functions of
$t$, $h$ that transform exactly linearly under \textsc{rg}
\cite{cardy.1996.scaling,aharony.1983.fields}. When studying the properties of
the Ising critical point, it is nearly always assumed that $\fF(X)$,
the universal scaling function, is an analytic function of $X$. However, it
has long been known that there exists an essential singularity in $\fF$
at $X=0$, though its effects have long been believed to be unobservable
\cite{fisher.1967.condensation}, or simply just neglected
\cite{guida.1997.3dising,schofield.1969.parametric,schofield.1969.correlation,caselle.2001.critical,josephson.1969.equation,fisher.1999.trigonometric}.
With careful analysis, we have found that assuming the presence of the
essential singularity is predictive of the scaling form of e.g. the
susceptibility.

The provenance of the essential singularity can be understood using the
methods of critical droplet theory for the decay of an Ising system in a
metastable state, i.e., an equilibrium Ising state for $T<T_\c$, $H>0$
subjected to a small negative external field $H<0$. The existence of an
essential singularity has also been suggested by transfer matrix
\cite{mccraw.1978.metastability,enting.1980.investigation} and \textsc{rg}
methods \cite{klein.1976.essential}.  It has long been known that the decay
rate $\Gamma$ of metastable states in statistical mechanics is often related
to the metastable free energy $F$ by $\Gamma\propto\im F$
\cite{langer.1969.metastable,penrose.1987.rigorous,gaveau.1989.analytic,privman.1982.analytic}.
`Metastable free energy' can be thought of as either an analytic continuation
of the free energy through the abrupt phase transition, or restriction of the
partition function trace to states in the vicinity of the local free energy
minimum that characterizes the metastable state. In any case, the free energy
develops a nonzero imaginary part in the metastable region. Heuristically,
this can be thought of as similar to what happens in quantum mechanics with a
non-unitary Hamiltonian: the imaginary part describes loss of probability in
the system that corresponds to decay. 

In critical droplet theory, the metastable state decays when a domain of the
equilibrium state forms whose surface-energy cost for growth is outweighed by
bulk-energy gains. There is numerical evidence that, near the critical point, droplets are spherical \cite{gunther.1993.transfer-matrix}. The free energy cost of the surface of a droplet
scales with its radius $R$ like $\Sigma S_\dim R^{\dim-1}$ and that of its
bulk scales like $-M|H|V_\dim R^\dim$, where $S_\dim$ and $V_\dim$ are the surface area and
volume of a $(\dim-1)$-sphere, respectively, and $\Sigma$ is the surface tension of the equilibrium--metastable interface. The critical droplet size then is
$R_\c=(\dim-1)\Sigma/M|H|$ and the free energy of the critical
droplet is $\Delta
F_\c=\pi^{\dim/2}\Sigma^\dim((\dim-1)/M|H|)^{\dim-1}/\Gamma(1+\dim/2)$.  
Assuming the typical singular scaling forms
$\Sigma/T=|t|^\mu\fS(h|t|^{-\Delta})$ and $M=|t|^\beta\mathcal
M(h|t|^{-\Delta})$ and using known hyperscaling relations
\cite{widom.1981.interface}, this implies a scaling form
\def\eqcritformone{
  T\frac{\pi^{\dim/2}(\dim-1)^{\dim-1}}{\Gamma(1+\dim/2)}\frac{\fS^\dim(h|t|^{-\beta\delta})}{(-h|t|^{-\beta\delta}\fM(h|t|^{-\beta\delta}))^{\dim-1}}
}
\def\eqcritformtwo{
  T\fG^{-(\dim-1)}(h|t|^{-\Delta})
}
\ifreprint
\[
  \begin{aligned}
    \Delta F_\c
    &=\eqcritformone\\
    &\sim\eqcritformtwo.
  \end{aligned}
\]
\else
\[
  \Delta F_\c=\eqcritformone\sim\eqcritformtwo.
\]
\fi
Since both surface tension and magnetization are finite and nonzero for $H=0$
at $T<T_\c$, $\fG(X)=-BX+\O(X^2)$ for small negative $X$ with
\[
  B=\frac{\fM(0)}{\dim-1}\bigg(\frac{\Gamma(1+\dim/2)}{\pi^{\dim/2}\fS(0)^\dim}\bigg)^{1/(\dim-1)}.
\]
This first term in the scaling function $\fG$ is related to the ratio between the correlation length $\xi$
and the critical domain radius $R_c$, with
\[
  Bh|t|^{-\beta\delta}=\bigg(\frac{\Gamma(1+\dim/2)}{\pi^{\dim/2}\fS(0)(\xi_0^-)^{\dim-1}}\bigg)^{1/(\dim-1)}\frac\xi{R_\c}
\]
where $\xi=\xi_0^-|t|^{-\nu}$ for $t<T_c$. Since $\fS(0)(\xi_0^-)^{\dim-1}$ is a
universal amplitude ratio, $\frac{Bh|t|^{-\beta\delta}}{\xi/R_c}$ is a
universal quantity.
% The constant $B$ should be universal near the critical point given careful
% definition of the variable $X$.
% \[
%   \begin{aligned}
%     \frac\xi{R_\c}
%     &=\frac{\xi_0^-\fM(0)}{(d-1)\mathcal
%     S(0)}h|t|^{-\beta\delta}
%     =\frac{(\xi_0^-/\xi_0^+)R_\chi
%     R_\xi^d}{(d-1)R_CR_\Sigma}\frac{h|t|^{-\beta\delta}}{\fM(0)^\delta D_\c}\\
%     &=C\frac{h}{D_\c}|\fM(0)^{1/\beta}t|^{-\beta\delta}
%   \end{aligned}
% \]
% \[
%   \frac BC=\bigg(\frac{\Gamma(1+\frac d2)}{\pi^{d/2}\fS(0)(\xi_0^-)^{d-1}}\bigg)^{1/(d-1)}
%   =\bigg(\frac{\Gamma(1+\frac d2)}{\pi^{d/2}R_\Sigma(\xi_0^-/\xi_0^+)^{d-1}}\bigg)^{1/(d-1)}
% \]
% $R_\Sigma=\fS(0)\xi_0^{d-1}$
% These are $R^+_\xi=\frac1{\sqrt{2\pi}}$, $R_\Sigma^+=1$
% $R_C=0.3185699$ $R_\chi=6.77828502$ $\xi_0^-/\xi_0^+=\frac12$
The decay rate of the metastable state is proportional to the Boltzmann factor
for the creation of a critical droplet, yielding
\[
  \im F\sim\Gamma\propto e^{-\beta\Delta F_\c}=e^{-\fG(h|t|^{-\beta\delta})^{-(\dim-1)}}.
\]
For $d>1$ this function has an essential singularity in the invariant
combination $h|t|^{-\beta\delta}$.

% $\Gamma/\Gamma_\sq=(D_\sq/D)(\fM(0)^{1/\beta}/\fM_\sq(0)^{1/\beta})^{-7/4}$

This form of $\im F$ for small $h$ is well known
\cite{langer.1967.condensation,harris.1984.metastability}.  We make the scaling
ansatz that the imaginary part of the metastable free energy has the same
singular behavior as the real part of the equilibrium free energy, and that for
small $t$, $h$, $\im F(t,h)=|t|^{2-\alpha}\fiF(h|t|^{-\beta\delta})$ for
\[
  \fiF(X)=-A\Theta(-X)(-X)^be^{-1/(-BX)^{\dim-1}},
  \label{eq:im.scaling}
\]
where $\Theta$ is the Heaviside function. Results from combining an analysis
of fluctuations on the surface of critical droplets with \textsc{rg} recursion
relations suggest that $b=-(d-3)d/2$ for $d=2,4$ and $b=-7/3$ for
$d=3$
\cite{houghton.1980.metastable,rudnick.1976.equations,gunther.1980.goldstone}.
Assuming that $F$ is analytic in the upper complex-$h$ plane, the real part of
$F$ in the equilibrium state can be extracted from this imaginary metastable
free energy using the Kramers--Kronig relation
\[
  \re F(t,h)=\frac1\pi\int_{-\infty}^\infty\frac{\im F(t,h')}{h'-h}\,\dd h'.
\]
This relationship has been used to compute high-order moments of the free
energy in $H$ in good agreement with transfer matrix expansions
\cite{lowe.1980.instantons}. Here, we compute the integral to come to explicit
functional forms.  In \threedee and \fourdee this can be computed
explicitly given our scaling ansatz, yielding
\def\eqthreedeeone{
  \fF^\threedee(Y/B)&=
  \frac{A}{12}\frac{e^{-1/Y^2}}{Y^2}
  \bigg[4Y\Gamma(\tfrac23)E_{5/3}(-Y^{-2})
}
\def\eqthreedeetwo{
  -\Gamma(\tfrac16)E_{7/6}(-Y^{-2})\bigg]
}
\def\eqfourdeeone{
  \fF^\fourdee(Y/B)&=
  \frac{A}{9\pi}\frac{e^{1/Y^3}}{Y^2}
  \Big[3\ei(-Y^{-3})
}
\def\eqfourdeetwo{
  +3\Gamma(\tfrac23)\Gamma(\tfrac13,Y^{-3})
  +\Gamma(\tfrac13)\Gamma(-\tfrac13,Y^{-3})\Big]
}
\ifreprint
\begin{align}
  &\begin{aligned}
    \eqthreedeeone\\
    &\hspace{6em}
    \eqthreedeetwo
  \end{aligned}
  \\
  &\begin{aligned}
    \eqfourdeeone
    \\
    &\hspace{2em}
    \eqfourdeetwo.
  \end{aligned}
\end{align}
\else
\begin{align}
  \eqthreedeeone\eqthreedeetwo
  \\
  \eqfourdeeone\eqfourdeetwo.
\end{align}
\fi
At the level of truncation we are working at, the Kramers--Kronig relation
does not converge in \twodee. However, the higher moments can still be
extracted, e.g., the susceptibility, by taking
\[
  \chi=\pd[2]Fh
  =\frac2\pi\int_{-\infty}^\infty\frac{\im F(t,h')}{(h'-h)^3}\,\dd h'.
\]
With $\chi=|t|^{-\gamma}\fX(h|t|^{-\Delta})$, this yields
\[
  \fX^\twodee(Y/B)=\frac{AB^2}{\pi Y^3}\big[Y(Y-1)-e^{1/Y}\ei(-1/Y)\big]
  \label{eq:sus_scaling}
\]
The constant $A$ can be fixed by zero-field results, with $\chi(t,0)|t|^\gamma=\lim_{X\to0}\fX^\twodee(X)=\frac{2AB^2}\pi$.
Scaling forms for the free energy can then be extracted by direct integration
and their constants of integration fixed by known zero field values, yielding
\begin{align}
  \label{eq:mag_scaling}
  \fM^\twodee(Y/B)
    &=\fM(0)+\frac{ABT_c}{\pi}\bigg(1-\frac{Y-1}Ye^{1/Y}\ei(-1/Y)\bigg)\\
  \fF^\twodee(Y/B)
    &=\fF(0)+T_cY\bigg(\frac{\fM(0)}B+\frac{AT_c}\pi e^{1/Y}\ei(-1/Y)\bigg)
\end{align}
with $F(t,h)=|t|^{2-\alpha}\fF(h|t|^{-\beta\delta})+t^{2-\alpha}\log|t|$ in two dimensions.

Previous work at zero field suggests that
$\fX(0)=\frac{2AB^2}\pi=C_{0-}/T_\c$, with $C_{0-}=0.025\,536\,971\,9$
\cite{barouch.1973.susceptibility}.  
Exact results for the \twodee Ising
model have $\fS(0)=4$ and $\fM(0)=(2^{5/2}\arcsinh1)^\beta$, so
$B=T_\c^2\fM(0)/\pi\fS(0)^2=(2^{27/16}\pi(\sinh^{-1}1)^{15/8})^{-1}$. 

How predictive are these scaling forms in the proximity of the critical point
and the abrupt transition line? We simulated the \twodee Ising model on square lattice using a form of the Wolff algorithm modified
to remain efficient in the presence of an external field. Briefly, the external field $H$ is applied by adding an extra spin $s_0$ with coupling $|H|$ to all others
\cite{dimitrovic.1991.finite}. A quickly converging estimate for the magnetization in the finite-size system was then made by taking $M=\sgn(H)s_0\sum s_i$, i.e., the magnetization relative to the external spin. For the \twodee Ising model on a square lattice, exact results at zero temperature have $\fS(0)=4/T_c$, $\fM(0)=(2^{5/2}\arcsinh1)^\beta$ \cite{onsager.1944.crystal}, and $\fX(0)=C_{0-}/T_\c$ with $C_{0-}=0.025\,536\,971\,9$
\cite{barouch.1973.susceptibility}, so that $B=\fM(0)/\pi\fS(0)^2=(2^{27/16}\pi(\arcsinh1)^{15/8})^{-1}$ and $A=\frac\pi2\fX(0)/B^2=2^{11/8}\pi^3(\arcsinh1)^{19/4}C_{0-}$. 
Data was then taken for susceptibility and
magnetization for $T_\c-T,H\leq0.1$. This data is plotted in
Fig.~\ref{fig:scaling_fits}, along with collapses of data onto a single universal curve
in the insets of those figures. As can be seen, there is very good agreement
between our proposed functional forms and what is measured.
However, there are systematic differences that can be seen most clearly in the
magnetization. Since our method is known to only be accurate for high moments
of the free energy, we should expect that low moments require corrections.
Therefore, we also fit those corrections of the form
\begin{align}
  \fX^{\twodee\prime}(X)&=\fX^\twodee(X)+\sum_{n=1}^Nf_n(BX)\\
  \fM^{\twodee\prime}(X)&=\fM^\twodee(X)+\frac{T_\c}B\sum_{n=1}^NF_n(BX)
\end{align}
where $F_n'(x)=f_n(x)$ and
\begin{align}
  f_n(x)&=\frac{C_nx^n}{1+(\lambda x)^{n+1}}\\
  F_n(x)&=\frac{C_n\lambda^{-(n+1)}}{n+1}\log(1+(\lambda x)^{n+1})
\end{align}
We fit these functions to our numeric data for $N=3$. The resulting curves are
also plotted in Fig.~\ref{fig:scaling_fits} as a dashed line.




%\begin{table}
%  \centering
%  \begin{tabular}{c|llc}
%    Lattice    & $T_\c$             & $\fM(0)^{1/\beta}$        & $D/D_\sq$   \\
%    \hline % -------------------------------------------------------------------
%    Square     & $2/\log(1+\sqrt2)$ & $2^{5/2}\arcsinh1$        & 1           \\
%    Triangular & $4/\log3$          & $4\log3$                  & $3^{3/2}/4$ \\
%    Hexagonal  & $2/\log(2+\sqrt3)$ & $\frac8{\sqrt3}\arccosh2$ & $3^{3/2}/8$
%  \end{tabular}
%  \caption{
%    The critical temperatures and amplitudes for the magetization along both the coexistence line and the critical isotherm, for three different lattices.
%  }
%  \label{tab:consts}
%\end{table}

\begin{figure}
  \input{figs/fig-susmag}
  \caption{
    Comparisons of scaling forms \eqref{eq:sus_scaling} and
    \eqref{eq:mag_scaling} to numeric data.  Data with
    sampling error taken from Monte Carlo simulations of a $4096\times4096$
    square-lattice Ising model with periodic boundary conditions and $T_\c-T=0.01,0.02,\ldots,0.1$ and
    $H=0.1\times(1,2^{-1/4},\ldots,2^{-50/4})$. The solid lines show our
    analytic results, while the dashed lines have polynomial corrections of
    the form \eqref{eq:poly} fit to the data for $N=3$, with $C_1=-0.00368$,
    $C_2=-0.0191$, $C_3=0.0350$, and $\lambda=2.42$.
  }
  \label{fig:scaling_fits}
\end{figure}

We have used results from the properties of the metastable Ising ferromagnet
and the analytic nature of the free energy to derive the universal scaling
functions for the free energy, and in \twodee the magnetization and
susceptibility, in the limit of small $t$ and $h$. Because of an essential
singularity in these functions at $h=0$---the abrupt transition line---their
form cannot be modified by analytic redefinition of control or thermodynamic
variables. These predictions match the results of simulations well. Having
demonstrated that the essential singularity in thermodynamic functions at the
abrupt singularity leads to observable effects. we hope that these functional
forms will be used in conjunction with traditional perturbation methods to
better express the equation of state of the Ising model in the whole of its
parameter space.

\begin{acknowledgments}
  The authors would like to thank Tom Lubensky, Andrea Liu, and Randy Kamien
  for helpful conversations. This work was partially supported by NSF grant
  DMR-1312160.
\end{acknowledgments}

\bibliography{essential-ising}

\end{document}

